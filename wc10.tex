\documentclass[a4paper]{exam}

\usepackage{amsmath,amssymb, amsthm}
\usepackage[a4paper]{geometry}
\usepackage{hyperref}
\usepackage{mdframed}

\title{Weekly Challenge 10: Turing Machine}
\author{CS 212 Nature of Computation\\Habib University}
\date{Fall 2023}

\theoremstyle{theorem}
\newtheorem{theorem}{Theorem}

\theoremstyle{claim}
\newtheorem{claim}{Claim}

\qformat{{\large\bf \thequestion. \thequestiontitle}\hfill}
\boxedpoints

\printanswers

\begin{document}
\maketitle

\begin{questions}

\titledquestion{Stay Put}

  A \textit{stay-put} Turing machine is similar to an ordinary Turing machine, but the transition function has the form
  \[
    \delta: Q\times \Gamma \to Q\times \Gamma \times \{\text{L, R, S} \}.
   \]
   At each point, the machine can move its head left or right, or let it stay in the same position.

   Prove or disprove the following claim,

   \begin{claim}
     The stay-put Turing machine is equivalent to the usual version.
   \end{claim}
   
  
\end{questions}
\end{document}

%%% Local Variables:
%%% mode: latex
%%% TeX-master: t
%%% End:
